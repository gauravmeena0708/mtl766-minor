\subsection*{Question 1}
\textbf{Question:} Explain the concept of a sample space in the context of multivariate analysis. How does it differ from the feature space?

\textbf{Solution:}
In multivariate analysis, the sample space is an $n$-dimensional space where $n$ is the number of observations. Each of the $p$ variables can be represented as a vector in this space. So, we have $p$ vectors in an $n$-dimensional space. This is in contrast to the feature space, which is a $p$-dimensional space where each of the $n$ observations is represented as a point.

\subsection*{Question 2}
\textbf{Question:} What is a vector projection? Provide the formula for projecting a vector $\mathbf{y}$ onto a vector $\mathbf{x}$.

\textbf{Solution:}
A vector projection of a vector $\mathbf{y}$ onto a vector $\mathbf{x}$ is the component of $\mathbf{y}$ that lies in the direction of $\mathbf{x}$. The formula is:
$$ \text{proj}_{\mathbf{x}} \mathbf{y} = \frac{\mathbf{y}^T \mathbf{x}}{\mathbf{x}^T \mathbf{x}} \mathbf{x} $$
This gives a vector in the direction of $\mathbf{x}$. The scalar value $\frac{\mathbf{y}^T \mathbf{x}}{\mathbf{x}^T \mathbf{x}}$ is the coordinate of the projection.

\subsection*{Question 3}
\textbf{Question:} Given vectors $\mathbf{y} = \begin{pmatrix} 3 \\ 4 \end{pmatrix}$ and $\mathbf{x} = \begin{pmatrix} 1 \\ 1 \end{pmatrix}$, find the projection of $\mathbf{y}$ onto $\mathbf{x}$.

\textbf{Solution:}
We use the formula for projection:
$$ \mathbf{y}^T \mathbf{x} = (3)(1) + (4)(1) = 7 $$
$$ \mathbf{x}^T \mathbf{x} = (1)^2 + (1)^2 = 2 $$
$$ \text{proj}_{\mathbf{x}} \mathbf{y} = \frac{7}{2} \mathbf{x} = \frac{7}{2} \begin{pmatrix} 1 \\ 1 \end{pmatrix} = \begin{pmatrix} 3.5 \\ 3.5 \end{pmatrix} $$

\subsection*{Question 4}
\textbf{Question:} How can we interpret the length of a vector in Euclidean space? What does the squared length of a mean-centered vector represent?

\textbf{Solution:}
The length of a vector $\mathbf{x} = (x_1, ..., x_n)^T$ is given by $L_{\mathbf{x}} = \sqrt{\sum_{i=1}^n x_i^2}$.
A mean-centered vector $\mathbf{d} = \mathbf{x} - \bar{x}\mathbf{1}$ has elements $d_i = x_i - \bar{x}$. The squared length of this vector is:
$$ L_{\mathbf{d}}^2 = \sum_{i=1}^n (x_i - \bar{x})^2 $$
This is $(n-1)$ times the sample variance of the variable $x$.

\subsection*{Question 5}
\textbf{Question:} Calculate the length of the vector $\mathbf{d} = \mathbf{y} - \bar{y}\mathbf{1}$, where $\mathbf{y} = \begin{pmatrix} 1 \\ 2 \\ 3 \end{pmatrix}$ and $\mathbf{1} = \begin{pmatrix} 1 \\ 1 \\ 1 \end{pmatrix}$.

\textbf{Solution:}
First, calculate the mean $\bar{y} = (1+2+3)/3 = 2$.
The mean-centered vector is:
$$ \mathbf{d} = \begin{pmatrix} 1 \\ 2 \\ 3 \end{pmatrix} - 2 \begin{pmatrix} 1 \\ 1 \\ 1 \end{pmatrix} = \begin{pmatrix} -1 \\ 0 \\ 1 \end{pmatrix} $$
The length of $\mathbf{d}$ is:
$$ L_{\mathbf{d}} = \sqrt{(-1)^2 + 0^2 + 1^2} = \sqrt{2} $$

\subsection*{Question 6}
\textbf{Question:} Define the cosine of the angle between two vectors. What do values of 1, 0, and -1 signify?

\textbf{Solution:}
The cosine of the angle $\theta$ between two vectors $\mathbf{x}$ and $\mathbf{y}$ is:
$$ \cos(\theta) = \frac{\mathbf{x}^T \mathbf{y}}{\|\mathbf{x}\| \|\mathbf{y}\|} $$
- $\cos(\theta) = 1$ means the vectors point in the same direction ($\theta=0^\circ$).
- $\cos(\theta) = 0$ means the vectors are orthogonal ($\theta=90^\circ$).
- $\cos(\theta) = -1$ means the vectors point in opposite directions ($\theta=180^\circ$).

\subsection*{Question 7}
\textbf{Question:} Find the cosine of the angle between the two vectors from Question 3, $\mathbf{y} = \begin{pmatrix} 3 \\ 4 \end{pmatrix}$ and $\mathbf{x} = \begin{pmatrix} 1 \\ 1 \end{pmatrix}$.

\textbf{Solution:}
$$ \mathbf{x}^T \mathbf{y} = 7 $$
$$ \|\mathbf{x}\| = \sqrt{1^2+1^2} = \sqrt{2} $$
$$ \|\mathbf{y}\| = \sqrt{3^2+4^2} = \sqrt{25} = 5 $$
$$ \cos(\theta) = \frac{7}{5\sqrt{2}} \approx 0.9899 $$

\subsection*{Question 8}
\textbf{Question:} How does the concept of cosine angle relate to the sample correlation coefficient?

\textbf{Solution:}
The sample correlation coefficient $r$ between two variables $x$ and $y$ is the cosine of the angle between their mean-centered vectors in the $n$-dimensional sample space. If $\mathbf{d}_x = \mathbf{x} - \bar{x}\mathbf{1}$ and $\mathbf{d}_y = \mathbf{y} - \bar{y}\mathbf{1}$, then:
$$ r_{xy} = \frac{\mathbf{d}_x^T \mathbf{d}_y}{\|\mathbf{d}_x\| \|\mathbf{d}_y\|} = \cos(\theta) $$

\subsection*{Question 9}
\textbf{Question:} Explain how a linear combination of variables can be viewed as a projection.

\textbf{Solution:}
Consider a linear combination of $p$ variables, $c_1 x_1 + ... + c_p x_p$. In the sample space, we have $p$ vectors $\mathbf{x}_1, ..., \mathbf{x}_p$. The linear combination forms a new vector $\mathbf{z} = c_1 \mathbf{x}_1 + ... + c_p \mathbf{x}_p$. This vector $\mathbf{z}$ lies in the subspace spanned by the original variable vectors. Each observation's value for this new variable, $z_i$, is its value on the new axis defined by the linear combination. This can be seen as a projection of the observation points onto this new axis.

\subsection*{Question 10}
\textbf{Question:} Project the first observation vector $\mathbf{x}_1 = \begin{pmatrix} 2 \\ 3 \end{pmatrix}$ from the feature space onto the vector $\mathbf{v} = \begin{pmatrix} 1 \\ -1 \end{pmatrix}$.

\textbf{Solution:}
This is a projection in the feature space.
$$ \mathbf{x}_1^T \mathbf{v} = (2)(1) + (3)(-1) = -1 $$
$$ \mathbf{v}^T \mathbf{v} = (1)^2 + (-1)^2 = 2 $$
$$ \text{proj}_{\mathbf{v}} \mathbf{x}_1 = \frac{-1}{2} \mathbf{v} = \begin{pmatrix} -0.5 \\ 0.5 \end{pmatrix} $$

\subsection*{Question 11}
\textbf{Question:} What is the geometric interpretation of the sample variance?

\textbf{Solution:}
Geometrically, the sample variance of a variable is proportional to the squared length of its mean-centered vector in the sample space. A larger variance means the vector is longer, indicating more spread in the data.
$$ s_x^2 = \frac{1}{n-1} \sum (x_i - \bar{x})^2 = \frac{1}{n-1} \|\mathbf{x} - \bar{x}\mathbf{1}\|^2 $$

\subsection*{Question 12}
\textbf{Question:} What is the geometric interpretation of the sample covariance?

\textbf{Solution:}
The sample covariance between two variables $x$ and $y$ is proportional to the dot product of their mean-centered vectors.
$$ s_{xy} = \frac{1}{n-1} (\mathbf{x} - \bar{x}\mathbf{1})^T (\mathbf{y} - \bar{y}\mathbf{1}) $$
The sign of the covariance is determined by the angle between these vectors. If the angle is less than 90 degrees, the covariance is positive. If it's greater than 90 degrees, it's negative.

\subsection*{Question 13}
\textbf{Question:} Given two mean-centered vectors $\mathbf{d}_1 = \begin{pmatrix} -1 \\ 0 \\ 1 \end{pmatrix}$ and $\mathbf{d}_2 = \begin{pmatrix} -1 \\ 1 \\ 0 \end{pmatrix}$, calculate their dot product. What does this imply about their sample covariance?

\textbf{Solution:}
$$ \mathbf{d}_1^T \mathbf{d}_2 = (-1)(-1) + (0)(1) + (1)(0) = 1 $$
Since the dot product is positive, the sample covariance between the two corresponding variables is positive. The sample covariance would be $1 / (n-1) = 1/2$.

\subsection*{Question 14}
\textbf{Question:} Describe how you would find a projection of a data set that maximizes the variance of the projected points.

\textbf{Solution:}
This is the core idea of Principal Component Analysis (PCA). We want to find a direction (a unit vector $\mathbf{a}$) such that when we project the data points onto this direction, the variance of the projected points is maximized. The projected values are given by $X\mathbf{a}$. The variance of these projected values is proportional to $\mathbf{a}^T S \mathbf{a}$, where $S$ is the covariance matrix. To maximize this quantity subject to $\|\mathbf{a}\|=1$, we find the eigenvector of $S$ corresponding to the largest eigenvalue. This eigenvector is the direction of maximum variance.

\subsection*{Question 15}
\textbf{Question:} If two vectors representing two variables are orthogonal in the sample space after being mean-centered, what does this imply about their correlation?

\textbf{Solution:}
If the mean-centered vectors $\mathbf{d}_x$ and $\mathbf{d}_y$ are orthogonal, their dot product is zero: $\mathbf{d}_x^T \mathbf{d}_y = 0$. Since the sample correlation is the cosine of the angle between these vectors, and the angle is 90 degrees, the correlation is $\cos(90^\circ) = 0$. This means the two variables are uncorrelated.

\subsection*{Question 16}
\textbf{Question:} Let $X_c$ be the $n \times p$ mean-centered data matrix. The projection of the $n$ observation points onto a p-dimensional vector $\mathbf{a}$ results in a new vector of data points $X_c \mathbf{a}$. Show that the sample variance of these projected points is given by $\mathbf{a}^T S \mathbf{a}$.

\textbf{Solution:}
The vector of projected points is $\mathbf{z} = X_c \mathbf{a}$. The mean of the projected points is $\bar{z}=0$.
The sample variance of the projected points is:
$$ s_z^2 = \frac{1}{n-1} \sum_{i=1}^n z_i^2 = \frac{1}{n-1} (X_c \mathbf{a})^T (X_c \mathbf{a}) = \mathbf{a}^T \left( \frac{1}{n-1} X_c^T X_c \right) \mathbf{a} = \mathbf{a}^T S \mathbf{a} $$
This result is fundamental to Principal Component Analysis (PCA).

\subsection*{Question 17}
\textbf{Question:} Let $\mathbf{d}_j$ and $\mathbf{d}_k$ be two mean-centered data vectors. The simple linear regression coefficient of variable $j$ on variable $k$ is $b_{jk} = \frac{\mathbf{d}_j^T \mathbf{d}_k}{\mathbf{d}_k^T \mathbf{d}_k}$. Interpret this geometrically.

\textbf{Solution:}
In the $n$-dimensional sample space, the vectors $\mathbf{d}_j$ and $\mathbf{d}_k$ represent the two variables. The formula for the regression coefficient $b_{jk}$ is identical to the scalar component of the projection of vector $\mathbf{d}_j$ onto vector $\mathbf{d}_k$.
The projection of $\mathbf{d}_j$ onto the line defined by $\mathbf{d}_k$ is:
$$ \text{proj}_{\mathbf{d}_k} \mathbf{d}_j = \left( \frac{\mathbf{d}_j^T \mathbf{d}_k}{\mathbf{d}_k^T \mathbf{d}_k} \right) \mathbf{d}_k = b_{jk} \mathbf{d}_k $$
The vector of predicted values, $\hat{\mathbf{d}}_j = b_{jk}\mathbf{d}_k$, is this projection.

\subsection*{Question 18}
\textbf{Question:} Consider the projection matrix $P = \frac{1}{n}\mathbf{1}\mathbf{1}^T$. Show that $P$ is idempotent and symmetric, and interpret its effect and the effect of $(I-P)$.

\textbf{Solution:}
$P$ is symmetric since $P^T = (\frac{1}{n}\mathbf{1}\mathbf{1}^T)^T = \frac{1}{n}\mathbf{1}\mathbf{1}^T = P$.
$P$ is idempotent since $P^2 = (\frac{1}{n}\mathbf{1}\mathbf{1}^T)(\frac{1}{n}\mathbf{1}\mathbf{1}^T) = \frac{1}{n^2}\mathbf{1}(\mathbf{1}^T\mathbf{1})\mathbf{1}^T = \frac{n}{n^2}\mathbf{1}\mathbf{1}^T = P$.
$P\mathbf{x} = \bar{x}\mathbf{1}$, so it projects a vector onto the mean vector.
$(I-P)\mathbf{x} = \mathbf{x} - \bar{x}\mathbf{1}$, so it produces the mean-centered vector.

\subsection*{Question 19}
\textbf{Question:} From a geometric perspective, what is the goal of Canonical Correlation Analysis (CCA)?

\textbf{Solution:}
Geometrically, CCA seeks to find a pair of vectors, one in the subspace spanned by the first set of variables and one in the subspace spanned by the second set, such that the angle between these two new vectors is minimized. The cosine of this minimum angle is the first canonical correlation.

\subsection*{Question 20}
\textbf{Question:} The CCA canonical vectors are found by solving $(S_{12}S_{22}^{-1}S_{21})\mathbf{a} = \lambda S_{11}\mathbf{a}$. Interpret the matrix $M = S_{12}S_{22}^{-1}S_{21}$.

\textbf{Solution:}
The matrix $M$ is the covariance matrix of the predicted values of the first set of variables, $\hat{X}_1$, when they are predicted from the second set of variables using multivariate multiple regression. It represents the component of the variance in the first set of variables that is explained by the second set.

\subsection*{Question 21}
\textbf{Question:} Justify that $(\mathbf{x}-\bar{\mathbf{x}})^{T}\mathbf{S}^{-1}(\mathbf{x}-\bar{\mathbf{x}})$ is a valid statistical distance measure that is scale-invariant.

\textbf{Solution:}
The squared Mahalanobis distance is a valid distance as it is non-negative, zero only if $\mathbf{x}=\bar{\mathbf{x}}$, and symmetric. It is scale-invariant because if we scale the data by $\mathbf{z}_i = D^{-1/2}\mathbf{x}_i$, the new covariance matrix is the correlation matrix $R$. The distance computed for $\mathbf{z}$ is $D_z^2 = (\mathbf{z}-\bar{\mathbf{z}})^{T}R^{-1}(\mathbf{z}-\bar{\mathbf{z}})$, which can be shown to be equal to the original distance $D_x^2$.

\subsection*{Question 22}
\textbf{Question:} Use the geometric interpretation in n-space to justify that generalised sample variance is a joint measure of variation. What is its major weakness?

\textbf{Solution:}
In n-space, $|S|$ is proportional to the squared volume of the parallelepiped formed by the $p$ mean-centered data vectors. This volume is large when the vectors are long (high variance) and not collinear (low correlation), so it captures joint variation. Its weakness is extreme sensitivity to the scale of the variables.

\subsection*{Question 23}
\textbf{Question:} (True/False) For any $\mathbf{X} \in \mathbb{R}^p$, the Mahalanobis distance $(\mathbf{X} - \boldsymbol{\mu})^T \boldsymbol{\Sigma}^{-1} (\mathbf{X} - \boldsymbol{\mu})$ is non-negative. Justify.

\textbf{Solution:}
True. Since $\boldsymbol{\Sigma}$ must be positive definite for its inverse to exist, $\boldsymbol{\Sigma}^{-1}$ is also positive definite. The quadratic form $\mathbf{v}^T \boldsymbol{\Sigma}^{-1} \mathbf{v}$ is therefore positive for any non-zero vector $\mathbf{v} = \mathbf{X} - \boldsymbol{\mu}$.

\subsection*{Question 24}
\textbf{Question:} Draw a constellation graph for the provided student marks data and identify the best, worst, and most average students.

\textbf{Solution:}
A constellation graph plots each observation as a star with rays proportional to the variable values.
Calculations show:
- Best student (furthest from origin): Student 9, with scores (4, 10, 8) and distance 13.42.
- Worst student (closest to origin): Student 4, with scores (2, 2, 3) and distance 4.12.
- Most average student (closest to the mean vector (2.55, 5.4, 5.3)): Student 3, with scores (2, 4, 5).

\subsection*{Question 25}
\textbf{Question:} In multiple regression, show that the vector of residuals $\mathbf{e} = (I-H)\mathbf{y}$ is orthogonal to the vector of fitted values $\hat{\mathbf{y}} = H\mathbf{y}$.

\textbf{Solution:}
We need to show $\hat{\mathbf{y}}^T \mathbf{e} = 0$.
$$ \hat{\mathbf{y}}^T \mathbf{e} = (H\mathbf{y})^T ((I-H)\mathbf{y}) = \mathbf{y}^T H^T (I-H) \mathbf{y} $$
The hat matrix $H$ is symmetric ($H^T=H$) and idempotent ($H^2=H$).
$$ \mathbf{y}^T H (I-H) \mathbf{y} = \mathbf{y}^T (H - H^2) \mathbf{y} = \mathbf{y}^T (\mathbf{0}) \mathbf{y} = 0 $$
Thus, the vectors are orthogonal.
