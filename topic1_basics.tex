\subsection*{Question 1}
\textbf{Question:} Define a multivariate data sample and the sample mean vector.

\textbf{Solution:}
A multivariate data sample consists of $n$ observations on $p$ variables. We can represent this data as a $n \times p$ matrix $X$, where $x_{ij}$ is the $i$-th observation of the $j$-th variable.
The sample mean vector is a $p \times 1$ vector $\bar{\mathbf{x}}$ where each element $\bar{x}_j$ is the average of the $n$ observations for the $j$-th variable.
$$ \bar{\mathbf{x}} = \frac{1}{n} \sum_{i=1}^{n} \mathbf{x}_i $$

\subsection*{Question 2}
\textbf{Question:} Given the following dataset with 2 variables:
$$ X = \begin{pmatrix} 2 & 3 \\ 4 & 5 \\ 6 & 7 \end{pmatrix} $$
Calculate the sample mean vector.

\textbf{Solution:}
The sample mean vector $\bar{\mathbf{x}}$ is calculated as:
$$ \bar{x}_1 = \frac{2+4+6}{3} = 4 $$
$$ \bar{x}_2 = \frac{3+5+7}{3} = 5 $$
So, the sample mean vector is:
$$ \bar{\mathbf{x}} = \begin{pmatrix} 4 \\ 5 \end{pmatrix} $$

\subsection*{Question 3}
\textbf{Question:} Define the sample variance-covariance matrix and the sample correlation matrix. Explain the relationship between them.

\textbf{Solution:}
The sample variance-covariance matrix, denoted by $S$, is a $p \times p$ symmetric matrix where the diagonal elements $s_{jj}$ are the variances of each variable and the off-diagonal elements $s_{jk}$ are the covariances between variables $j$ and $k$.
$$ S = \frac{1}{n-1} \sum_{i=1}^{n} (\mathbf{x}_i - \bar{\mathbf{x}})(\mathbf{x}_i - \bar{\mathbf{x}})^T $$
The sample correlation matrix, $R$, is a $p \times p$ matrix where the elements $r_{jk}$ are the sample correlation coefficients between variables $j$ and $k$.
$$ r_{jk} = \frac{s_{jk}}{\sqrt{s_{jj}}\sqrt{s_{kk}}} $$
The relationship is $R = D^{-1/2} S D^{-1/2}$, where $D$ is a diagonal matrix of the variances from $S$.

\subsection*{Question 4}
\textbf{Question:} For the dataset in Question 2, calculate the sample variance-covariance matrix.

\textbf{Solution:}
First, we calculate the deviations from the mean:
$$ X - \bar{\mathbf{x}} = \begin{pmatrix} 2-4 & 3-5 \\ 4-4 & 5-5 \\ 6-4 & 7-5 \end{pmatrix} = \begin{pmatrix} -2 & -2 \\ 0 & 0 \\ 2 & 2 \end{pmatrix} $$
The sum of squared products matrix is:
$$ (X - \bar{\mathbf{x}})^T (X - \bar{\mathbf{x}}) = \begin{pmatrix} (-2)^2+0^2+2^2 & (-2)(-2)+0(0)+2(2) \\ (-2)(-2)+0(0)+2(2) & (-2)^2+0^2+2^2 \end{pmatrix} = \begin{pmatrix} 8 & 8 \\ 8 & 8 \end{pmatrix} $$
The variance-covariance matrix $S$ is this matrix divided by $n-1 = 2$:
$$ S = \frac{1}{2} \begin{pmatrix} 8 & 8 \\ 8 & 8 \end{pmatrix} = \begin{pmatrix} 4 & 4 \\ 4 & 4 \end{pmatrix} $$

\subsection*{Question 5}
\textbf{Question:} From the result of Question 4, calculate the sample correlation matrix.

\textbf{Solution:}
We have $s_{11}=4$, $s_{22}=4$, and $s_{12}=4$.
$$ r_{11} = r_{22} = 1 $$
$$ r_{12} = \frac{s_{12}}{\sqrt{s_{11}s_{22}}} = \frac{4}{\sqrt{4 \cdot 4}} = \frac{4}{4} = 1 $$
The correlation matrix is:
$$ R = \begin{pmatrix} 1 & 1 \\ 1 & 1 \end{pmatrix} $$

\subsection*{Question 6}
\textbf{Question:} What is a feature space? How do you visualize a multivariate data sample in it?

\textbf{Solution:}
A feature space is a $p$-dimensional space where each dimension corresponds to one of the $p$ variables (features) of the dataset. Each observation $\mathbf{x}_i$ is represented as a point in this space. For $p=2$ or $p=3$, we can create a scatter plot of the $n$ points. For $p>3$, we can use techniques like scatter plot matrices to visualize pairs of variables.

\subsection*{Question 7}
\textbf{Question:} Define the Mahalanobis distance. How does it differ from the Euclidean distance?

\textbf{Solution:}
The Mahalanobis distance between two points $\mathbf{x}$ and $\mathbf{y}$ in a $p$-dimensional space with covariance matrix $S$ is:
$$ D_M(\mathbf{x}, \mathbf{y}) = \sqrt{(\mathbf{x}-\mathbf{y})^T S^{-1} (\mathbf{x}-\mathbf{y})} $$
It differs from Euclidean distance by accounting for the covariance among variables. It is scale-invariant and corrects for correlation.

\subsection*{Question 8}
\textbf{Question:} Given a mean vector $\bar{\mathbf{x}} = \begin{pmatrix} 0 \\ 0 \end{pmatrix}$, a covariance matrix $S = \begin{pmatrix} 1 & 0.5 \\ 0.5 & 1 \end{pmatrix}$, and a point $\mathbf{x}_0 = \begin{pmatrix} 1 \\ 1 \end{pmatrix}$, calculate the Mahalanobis distance from $\mathbf{x}_0$ to the mean.

\textbf{Solution:}
First, find the inverse of $S$:
$$ S^{-1} = \frac{1}{1-0.25} \begin{pmatrix} 1 & -0.5 \\ -0.5 & 1 \end{pmatrix} = \frac{4}{3} \begin{pmatrix} 1 & -0.5 \\ -0.5 & 1 \end{pmatrix} $$
The Mahalanobis distance is:
$$ D_M^2 = (\mathbf{x}_0-\bar{\mathbf{x}})^T S^{-1} (\mathbf{x}_0-\bar{\mathbf{x}}) = \begin{pmatrix} 1 & 1 \end{pmatrix} \frac{4}{3} \begin{pmatrix} 1 & -0.5 \\ -0.5 & 1 \end{pmatrix} \begin{pmatrix} 1 \\ 1 \end{pmatrix} = \frac{4}{3} \begin{pmatrix} 1 & 1 \end{pmatrix} \begin{pmatrix} 0.5 \\ 0.5 \end{pmatrix} = \frac{4}{3}(0.5+0.5) = \frac{4}{3} $$
$$ D_M = \sqrt{4/3} \approx 1.1547 $$

\subsection*{Question 9}
\textbf{Question:} Describe the geometric shape of points that are at a constant statistical distance from the mean.

\textbf{Solution:}
The set of points $\mathbf{x}$ that are at a constant Mahalanobis distance $c$ from the mean vector $\bar{\mathbf{x}}$ forms an ellipsoid in the $p$-dimensional space. The equation for this ellipsoid is:
$$ (\mathbf{x}-\bar{\mathbf{x}})^T S^{-1} (\mathbf{x}-\bar{\mathbf{x}}) = c^2 $$
The center of the ellipsoid is $\bar{\mathbf{x}}$, and its axes are determined by the eigenvectors and eigenvalues of the covariance matrix $S$.

\subsection*{Question 10}
\textbf{Question:} Write down the equation for an ellipsoid of constant statistical distance for a 2-dimensional case with a diagonal covariance matrix $S = \begin{pmatrix} s_{11} & 0 \\ 0 & s_{22} \end{pmatrix}$.

\textbf{Solution:}
For a diagonal covariance matrix, $S^{-1} = \begin{pmatrix} 1/s_{11} & 0 \\ 0 & 1/s_{22} \end{pmatrix}$.
The equation for the ellipsoid is:
$$ (\mathbf{x}-\bar{\mathbf{x}})^T S^{-1} (\mathbf{x}-\bar{\mathbf{x}}) = \frac{(x_1-\bar{x}_1)^2}{s_{11}} + \frac{(x_2-\bar{x}_2)^2}{s_{22}} = c^2 $$
This is the standard equation of an ellipse centered at $(\bar{x}_1, \bar{x}_2)$ with axes parallel to the coordinate axes.

\subsection*{Question 11}
\textbf{Question:} Define the total variance and the generalized variance. What does a generalized variance of zero imply?

\textbf{Solution:}
The total variance is the sum of the variances of all variables, which is the trace of the covariance matrix $S$.
$$ \text{Total Variance} = \text{tr}(S) = \sum_{j=1}^{p} s_{jj} $$
The generalized variance is the determinant of the covariance matrix $S$.
$$ \text{Generalized Variance} = |S| $$
A generalized variance of zero implies that the covariance matrix is singular, which means there is at least one linear dependency among the variables.

\subsection*{Question 12}
\textbf{Question:} Calculate the total variance and generalized variance for the covariance matrix from Question 4.

\textbf{Solution:}
The covariance matrix is $S = \begin{pmatrix} 4 & 4 \\ 4 & 4 \end{pmatrix}$.
Total variance = tr(S) = 4 + 4 = 8.
Generalized variance = |S| = (4)(4) - (4)(4) = 0.
The zero generalized variance indicates that the variables are perfectly correlated.

\subsection*{Question 13}
\textbf{Question:} Show that the sample correlation matrix is symmetric and has 1s on the diagonal.

\textbf{Solution:}
The correlation coefficient $r_{jk} = \frac{s_{jk}}{\sqrt{s_{jj}s_{kk}}}$.
Symmetry: $r_{kj} = \frac{s_{kj}}{\sqrt{s_{kk}s_{jj}}} = \frac{s_{jk}}{\sqrt{s_{jj}s_{kk}}} = r_{jk}$ since $s_{kj}=s_{jk}$.
Diagonal elements: $r_{jj} = \frac{s_{jj}}{\sqrt{s_{jj}s_{jj}}} = \frac{s_{jj}}{s_{jj}} = 1$.

\subsection*{Question 14}
\textbf{Question:} If a variable is added to a dataset, what happens to the dimensions of the sample mean vector and sample variance-covariance matrix?

\textbf{Solution:}
If we add a variable, the number of variables $p$ becomes $p+1$.
The sample mean vector $\bar{\mathbf{x}}$ will have its dimension increase from $p \times 1$ to $(p+1) \times 1$.
The sample variance-covariance matrix $S$ will have its dimensions increase from $p \times p$ to $(p+1) \times (p+1)$.

\subsection*{Question 15}
\textbf{Question:} Consider a dataset with 3 variables. The covariance matrix is given by:
$$ S = \begin{pmatrix} 25 & -2 & 4 \\ -2 & 4 & 1 \\ 4 & 1 & 9 \end{pmatrix} $$
Find the correlation between variable 1 and 3.

\textbf{Solution:}
We need to find $r_{13}$. We have $s_{13}=4$, $s_{11}=25$, and $s_{33}=9$.
$$ r_{13} = \frac{s_{13}}{\sqrt{s_{11}s_{33}}} = \frac{4}{\sqrt{25 \cdot 9}} = \frac{4}{\sqrt{225}} = \frac{4}{15} \approx 0.2667 $$
The correlation between variable 1 and 3 is approximately 0.2667.

\subsection*{Question 16}
\textbf{Question:} Prove that the sample generalized variance, $|S|$, is zero if and only if the columns of the mean-centered data matrix $X_c$ are linearly dependent.

\textbf{Solution:}
The sample covariance matrix is given by $S = \frac{1}{n-1} X_c^T X_c$, where $X_c$ is the $n \times p$ mean-centered data matrix.
The generalized variance is $|S| = |\frac{1}{n-1} X_c^T X_c| = (\frac{1}{n-1})^p |X_c^T X_c|$.
Thus, $|S| = 0$ if and only if $|X_c^T X_c| = 0$.
The matrix $X_c^T X_c$ is a $p \times p$ matrix. Its determinant is zero if and only if the matrix is singular.
A matrix $A$ is singular if and only if there exists a non-zero vector $\mathbf{v}$ such that $A\mathbf{v} = \mathbf{0}$.
So, $|X_c^T X_c| = 0$ if and only if there exists a non-zero $p \times 1$ vector $\mathbf{a}$ such that $X_c^T X_c \mathbf{a} = \mathbf{0}$.
Multiplying by $\mathbf{a}^T$ on the left, we get $\mathbf{a}^T X_c^T X_c \mathbf{a} = 0$, which is $\|X_c \mathbf{a}\|^2 = 0$.
This implies $X_c \mathbf{a} = \mathbf{0}$.
$X_c \mathbf{a}$ is a linear combination of the columns of $X_c$. If $X_c \mathbf{a} = \mathbf{0}$ for a non-zero $\mathbf{a}$, this is the definition of the columns of $X_c$ being linearly dependent.
Conversely, if the columns of $X_c$ are linearly dependent, there exists a non-zero $\mathbf{a}$ such that $X_c \mathbf{a} = \mathbf{0}$, which implies $X_c^T X_c \mathbf{a} = \mathbf{0}$, which means $X_c^T X_c$ is singular and its determinant is zero. Therefore, $|S|=0$.

\subsection*{Question 17}
\textbf{Question:} Prove that the sample covariance matrix $S$ is positive semi-definite. That is, show $\mathbf{a}^T S \mathbf{a} \ge 0$ for any constant vector $\mathbf{a}$. When does strict equality, $\mathbf{a}^T S \mathbf{a} = 0$, hold?

\textbf{Solution:}
Let $\mathbf{a}$ be any $p \times 1$ constant vector. Consider the quadratic form $\mathbf{a}^T S \mathbf{a}$.
$$ \mathbf{a}^T S \mathbf{a} = \mathbf{a}^T \left( \frac{1}{n-1} \sum_{i=1}^n (\mathbf{x}_i - \bar{\mathbf{x}})(\mathbf{x}_i - \bar{\mathbf{x}})^T \right) \mathbf{a} $$
$$ = \frac{1}{n-1} \sum_{i=1}^n \mathbf{a}^T (\mathbf{x}_i - \bar{\mathbf{x}})(\mathbf{x}_i - \bar{\mathbf{x}})^T \mathbf{a} $$
Let $y_i = (\mathbf{x}_i - \bar{\mathbf{x}})^T \mathbf{a}$, which is a scalar. Then $\mathbf{a}^T (\mathbf{x}_i - \bar{\mathbf{x}})$ is also a scalar, $y_i$. So the expression becomes:
$$ = \frac{1}{n-1} \sum_{i=1}^n y_i^2 $$
Since $y_i^2 \ge 0$, the sum is non-negative. Thus, $\mathbf{a}^T S \mathbf{a} \ge 0$, and $S$ is positive semi-definite.

Strict equality, $\mathbf{a}^T S \mathbf{a} = 0$, holds if and only if $\sum y_i^2 = 0$, which means $y_i = 0$ for all $i=1, \dots, n$.
$y_i = (\mathbf{x}_i - \bar{\mathbf{x}})^T \mathbf{a} = 0$ for all $i$. This means the linear combination of the mean-centered variables, represented by vector $\mathbf{a}$, is zero for all observations. This corresponds to the columns of the mean-centered data matrix $X_c$ being linearly dependent (as shown in Q16), where $\mathbf{a}$ is the vector of coefficients of the linear combination.

\subsection*{Question 18}
\textbf{Question:} Show that the total sample variance and the sample generalized variance are invariant under a rotation of the data. A rotation is a linear transformation $\mathbf{y}_i = R\mathbf{x}_i$ where $R$ is an orthogonal matrix ($R^T R = I$) with determinant $|R|=1$.

\textbf{Solution:}
Let the original data be $\mathbf{x}_i$ with sample mean $\bar{\mathbf{x}}$ and sample covariance $S_x$.
The transformed data is $\mathbf{y}_i = R\mathbf{x}_i$.
The new sample mean is $\bar{\mathbf{y}} = \frac{1}{n}\sum R\mathbf{x}_i = R(\frac{1}{n}\sum\mathbf{x}_i) = R\bar{\mathbf{x}}$.
The new sample covariance matrix $S_y$ is:
$$ S_y = \frac{1}{n-1} \sum (\mathbf{y}_i - \bar{\mathbf{y}})(\mathbf{y}_i - \bar{\mathbf{y}})^T = \frac{1}{n-1} \sum (R\mathbf{x}_i - R\bar{\mathbf{x}})(R\mathbf{x}_i - R\bar{\mathbf{x}})^T $$
$$ = \frac{1}{n-1} \sum R(\mathbf{x}_i - \bar{\mathbf{x}})(\mathbf{x}_i - \bar{\mathbf{x}})^T R^T = R \left( \frac{1}{n-1} \sum (\mathbf{x}_i - \bar{\mathbf{x}})(\mathbf{x}_i - \bar{\mathbf{x}})^T \right) R^T = R S_x R^T $$
\textbf{Total Variance:}
The new total variance is $\text{tr}(S_y) = \text{tr}(R S_x R^T)$. Using the cyclic property of the trace, $\text{tr}(ABC) = \text{tr}(CAB)$:
$$ \text{tr}(S_y) = \text{tr}(R^T R S_x) = \text{tr}(I S_x) = \text{tr}(S_x) $$
So the total variance is invariant under rotation.

\textbf{Generalized Variance:}
The new generalized variance is $|S_y| = |R S_x R^T|$. Using the property $|ABC| = |A||B||C|$:
$$ |S_y| = |R| |S_x| |R^T| = |R| |S_x| |R| = |R|^2 |S_x| $$
For a rotation matrix, $|R|=1$, so $|R|^2=1$.
$$ |S_y| = |S_x| $$
So the generalized variance is also invariant under rotation.

\subsection*{Question 19}
\textbf{Question:} Let $S$ be a positive definite sample covariance matrix. It has a unique Cholesky decomposition $S = LL^T$, where $L$ is a lower triangular matrix with positive diagonal elements. Explain how this decomposition can be used to simulate multivariate data that exhibits a specific covariance structure.

\textbf{Solution:}
The Cholesky decomposition is a powerful tool for simulating multivariate data with a desired correlation structure. The process is as follows:
1. Start with a vector $\mathbf{z} = (z_1, \dots, z_p)^T$ of $p$ independent standard normal random variables (i.e., $z_j \sim N(0,1)$). The mean of this vector is $E[\mathbf{z}] = \mathbf{0}$ and its covariance matrix is $\text{Cov}(\mathbf{z}) = I$, the identity matrix.
2. Let the desired covariance structure for the simulated data be given by the matrix $S$, and let the desired mean be $\bar{\mathbf{x}}$.
3. First, find the Cholesky decomposition of the target covariance matrix, $S = LL^T$.
4. Transform the vector of independent standard normal variables $\mathbf{z}$ using the lower triangular matrix $L$ and the target mean $\bar{\mathbf{x}}$:
$$ \mathbf{x} = \bar{\mathbf{x}} + L\mathbf{z} $$
The resulting vector $\mathbf{x}$ will have the desired properties. Let's check its mean and covariance:
$$ E[\mathbf{x}] = E[\bar{\mathbf{x}} + L\mathbf{z}] = \bar{\mathbf{x}} + L E[\mathbf{z}] = \bar{\mathbf{x}} + L\mathbf{0} = \bar{\mathbf{x}} $$
$$ \text{Cov}(\mathbf{x}) = \text{Cov}(\bar{\mathbf{x}} + L\mathbf{z}) = L \text{Cov}(\mathbf{z}) L^T = L I L^T = LL^T = S $$
This procedure generates a random vector $\mathbf{x}$ from a distribution with the specified sample mean and sample covariance matrix $S$. Repeating this process $n$ times would generate a full dataset. This technique is fundamental in Monte Carlo simulations for risk management, financial modeling, and statistical power analysis.

\subsection*{Question 20}
\textbf{Question:} The sample correlation matrix $R$ is related to the covariance matrix $S$ by $S = D^{1/2} R D^{1/2}$, where $D$ is the diagonal matrix of sample variances $s_{jj}$. Prove that the generalized variance $|S|$ can be expressed as the product of the individual sample variances and the determinant of the correlation matrix, i.e., $|S| = (\prod_{j=1}^p s_{jj}) |R|$. What does this imply about the relationship between inter-correlation among variables and the generalized variance?

\textbf{Solution:}
We start with the relationship $S = D^{1/2} R D^{1/2}$.
Taking the determinant of both sides:
$$ |S| = |D^{1/2} R D^{1/2}| $$
Using the property that $|ABC| = |A||B||C|$:
$$ |S| = |D^{1/2}| |R| |D^{1/2}| = |D^{1/2}|^2 |R| $$
The matrix $D^{1/2}$ is a diagonal matrix with elements $\sqrt{s_{jj}}$ on the diagonal. The determinant of a diagonal matrix is the product of its diagonal elements.
$$ |D^{1/2}| = \prod_{j=1}^p \sqrt{s_{jj}} $$
Therefore, $|D^{1/2}|^2 = \left(\prod_{j=1}^p \sqrt{s_{jj}}\right)^2 = \prod_{j=1}^p s_{jj}$.
Substituting this back into the equation for $|S|$:
$$ |S| = \left(\prod_{j=1}^p s_{jj}\right) |R| $$
This completes the proof.

\textbf{Implication:}
This result shows that the generalized variance $|S|$ is a combination of the total individual variances (product of $s_{jj}$) and the inter-correlations among the variables (captured by $|R|$).
The determinant of the correlation matrix, $|R|$, ranges from 0 to 1.
- If the variables are uncorrelated, $R=I$ and $|R|=1$. In this case, $|S| = \prod s_{jj}$, and the generalized variance is maximized for a given set of individual variances.
- If the variables are perfectly correlated (linearly dependent), then $|R|=0$, which implies $|S|=0$.
- As the magnitude of correlation among variables increases, $|R|$ decreases from 1 towards 0, which in turn reduces the generalized variance $|S|$. Therefore, higher inter-correlation leads to a smaller "effective" volume of the data cloud in the feature space, as measured by the generalized variance.
